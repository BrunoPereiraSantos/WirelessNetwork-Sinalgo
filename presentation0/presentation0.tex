
%%%%%%%%%%%%%%%%%%%%%%%%%%%%%%%%%%%%%%%%%
% Beamer Presentation
% LaTeX Template
% Version 1.0 (10/11/12)
%
% This template has been downloaded from:
% http://www.LaTeXTemplates.com
%
% License:
% CC BY-NC-SA 3.0 (http://creativecommons.org/licenses/by-nc-sa/3.0/)
%
%%%%%%%%%%%%%%%%%%%%%%%%%%%%%%%%%%%%%%%%%

%----------------------------------------------------------------------------------------
%	PACKAGES AND THEMES
%----------------------------------------------------------------------------------------

\documentclass{beamer}


\usepackage{graphicx} % Allows including images
\usepackage{booktabs} % Allows the use of \toprule, \midrule and \bottomrule in tables
\usepackage[brazil]{babel}   
\usepackage[utf8]{inputenc}  
\usepackage{caption}  %Allows including capitons on images
%\usepackage{subcaption} %allow including multiples figures side to side
\usepackage{ragged2e} %allows justify text in tables
\usepackage{ulem} %allows striker text
\usepackage{subfigure}
\usepackage{hyperref}
\usepackage{color}
\usepackage{ragged2e}
\usepackage{listings}


\mode<presentation> {

% The Beamer class comes with a number of default slide themes
% which change the colors and layouts of slides. Below this is a list
% of all the themes, uncomment each in turn to see what they look like.

%\usetheme{default}
%\usetheme{AnnArbor}
%\usetheme{Antibes}
%\usetheme{Bergen}
%\usetheme{Berkeley}
%\usetheme{Berlin}
%\usetheme{Boadilla}
%\usetheme{CambridgeUS}
%\usetheme{Copenhagen}
\usetheme{Darmstadt}
%\usetheme{Dresden}
%\usetheme{Frankfurt}
%\usetheme{Goettingen}
%\usetheme{Hannover}
%\usetheme{Ilmenau}
%\usetheme{JuanLesPins}
%\usetheme{Luebeck}
%\usetheme{Madrid}
%\usetheme{Malmoe}
%\usetheme{Marburg}
%\usetheme{Montpellier}
%\usetheme{PaloAlto}
%\usetheme{Pittsburgh}
%\usetheme{Rochester}
%\usetheme{Singapore}
%\usetheme{Szeged}
%\usetheme{Warsaw}

% As well as themes, the Beamer class has a number of color themes
% for any slide theme. Uncomment each of these in turn to see how it
% changes the colors of your current slide theme.

%\usecolortheme{albatross}
%\usecolortheme{beaver}
%\usecolortheme{beetle}
%\usecolortheme{crane}
%\usecolortheme{dolphin}
%\usecolortheme{dove}
%\usecolortheme{fly}
%\usecolortheme{lily}
%\usecolortheme{orchid}
\usecolortheme{rose}
%\usecolortheme{seagull}
%\usecolortheme{seahorse}
%\usecolortheme{whale}
%\usecolortheme{wolverine}


% Set Font

%\usefonttheme{default}
%\usefonttheme{professionalfonts}
%\usefonttheme{serif}
%\usefonttheme{structurebold}
%\usefonttheme{structureitalicserif}

%\setbeamertemplate{footline} % To remove the footer line in all slides uncomment this line
%\setbeamertemplate{footline}[page number] % To replace the footer line in all slides with a simple slide count uncomment this line

%\setbeamertemplate{navigation symbols}{} % To remove the navigation symbols from the bottom of all slides uncomment this line

}



\definecolor{mygreen}{rgb}{0,0.6,0}
\definecolor{mygray}{rgb}{0.5,0.5,0.5}
\definecolor{mymauve}{rgb}{0.58,0,0.82}

\lstset{ %
  backgroundcolor=\color[rgb]{0.98, 1, 0.75},   % choose the background color; you must add \usepackage{color} or \usepackage{xcolor}
  basicstyle=\ttfamily\scriptsize,        % the size of the fonts that are used for the code
  breakatwhitespace=false,         % sets if automatic breaks should only happen at whitespace
  breaklines=true,                 % sets automatic line breaking
  captionpos=b,                    % sets the caption-position to bottom
  commentstyle=\color{mygreen},    % comment style
  deletekeywords={...},            % if you want to delete keywords from the given language
  escapeinside={\%*}{*)},          % if you want to add LaTeX within your code
  extendedchars=true,              % lets you use non-ASCII characters; for 8-bits encodings only, does not work with UTF-8
  frame=single,                    % adds a frame around the code
  keepspaces=true,                 % keeps spaces in text, useful for keeping indentation of code (possibly needs columns=flexible)
  keywordstyle=\color{blue},       % keyword style
  language=Java,                 % the language of the code
  morekeywords={*,...},            % if you want to add more keywords to the set
  numbers=left,                    % where to put the line-numbers; possible values are (none, left, right)
  numbersep=5pt,                   % how far the line-numbers are from the code
  numberstyle=\tiny\color{mygray}, % the style that is used for the line-numbers
  rulecolor=\color{black},         % if not set, the frame-color may be changed on line-breaks within not-black text (e.g. comments (green here))
  showspaces=false,                % show spaces everywhere adding particular underscores; it overrides 'showstringspaces'
  showstringspaces=false,          % underline spaces within strings only
  showtabs=false,                  % show tabs within strings adding particular underscores
  stepnumber=2,                    % the step between two line-numbers. If it's 1, each line will be numbered
  stringstyle=\color{mymauve},     % string literal style
  tabsize=2,                       % sets default tabsize to 2 spaces
  title=\lstname                   % show the filename of files included with \lstinputlisting; also try caption instead of title
}

%----------------------------------------------------------------------------------------
%	TITLE PAGE
%----------------------------------------------------------------------------------------

\title[Simuladores de rede Sem Fio]{Simuladores de rede sem fio} % The short title appears at the bottom of every slide, the full title is only on the title page

\author{Bruno Pereira} % Your name
\institute[UFMG] % Your institution as it will appear on the bottom of every slide, may be shorthand to save space
{
Universidade Federal de Minas Gerais \\ % Your institution for the title page
\medskip
\textit{bruno.ps@dcc.ufmg.com} % Your email address
}
\date{\tiny\today} % Date, can be changed to a custom date

\begin{document}


\begin{frame}
\titlepage % Print the title page as the first slide
\end{frame}

\begin{frame}
\frametitle{Agenda} % Table of contents slide, comment this block out to remove it
\tableofcontents % Throughout your presentation, if you choose to use \section{} and \subsection{} commands, these will automatically be printed on this slide as an overview of your presentation
\end{frame}

%----------------------------------------------------------------------------------------
%	PRESENTATION SLIDES
%----------------------------------------------------------------------------------------

%------------------------------------------------
\section{Introdução} % Sections can be created in order to organize your presentation into discrete blocks, all sections and subsections are automatically printed in the table of contents as an overview of the talk
%------------------------------------------------

\subsection{Simulação de rede} % A subsection can be created just before a set of slides with a common theme to further break down your presentation into chunks

\begin{frame}\frametitle{Simulação de rede de computadores}

\begin{block}{O que é?}
\begin{itemize}
	\item É uma técnica onde o programa (software) modela o comportamento de uma rede.
	\item Esse programa modela a interação entre as diferentes entidades de uma rede.
	\begin{itemize}
		\item Roteadores, Switches, Nodes, Pontos de Acesso, Enlaces ...
	\end{itemize}
	\item A maioria dos simuladores de redes utilizam simulação baseada em eventos discretos
	\begin{itemize}
		\item O estado das variáveis mudam em momentos discretos do tempo
	\end{itemize}
\end{itemize}
\end{block}

\end{frame}


%------------------------------------------------

\begin{frame}\frametitle{Simulador de redes}

\begin{block}{O que é?}
\begin{itemize}
	\item Um simulador de rede é um programa que prediz o comportamento da rede.
	\begin{itemize}
		\item Simuladores de redes passaram a ser utilizados, pois a comunicação em redes se tornaram complexas para métodos de análises tradicionais
	\end{itemize}
	\item Nos simuladores, a rede é modelada através:
	\begin{itemize}
		\item Dispositivos, links, aplicações.
		\item E objetiva-se medir o desempenho e correta operação dos mesmos em diferentes cerários.
	\end{itemize}
	\item Hoje os simuladores suportam diversas tecnologias de rede:
	\begin{itemize}
		\item Wireless LANs, Mobile/Vehicular Ad Hoc Networks, WSN, cognitive radios, Rede celular (5G), IoT, etc.
	\end{itemize}
\end{itemize}
\end{block}

\end{frame}

%------------------------------------------------

\begin{frame}\frametitle{Simulações}

\begin{block}{Características das simulações}
	\begin{itemize}
		\item Simuladores podem ser construídos orientados a GUI or CLI
		\item O modelo/configuração do simualdor de rede descreve:
		\begin{itemize}
			\item A rede
			\begin{itemize}
				\item Nós, roteadores, switches, enlaces
			\end{itemize}
			\item Os eventos
			\begin{itemize}
				\item Transmissões de dados, erros nos pacotes, mobilidade, etc.
			\end{itemize}
		\end{itemize}
		
		\item Como saída tem-se:
		\begin{itemize}
			\item Métricas ao nível de rede
			\begin{itemize}
				\item links, devices, etc.
			\end{itemize}
			\item Rastros (traces) de simulação
			\begin{itemize}
				\item Arquivos de traces de cada pacote, cada evento que ocorreu
			\end{itemize}
		\end{itemize}
	\end{itemize}
\end{block}

\end{frame}
%------------------------------------------------

\begin{frame}\frametitle{Simulações}

\begin{block}{Características das simulações}
	\begin{itemize}
		\item Dirigidos a eventos discretos: a maioria dos simualdores são desse tipo
		\begin{itemize}
			\item As entidades presentes nas simulação reagem aos eventos
			\item Uma lista de ``eventos'' pendentes é mantida e são processados em ordem
			\item Um evento pode engatilhar outros eventos futuros
		\end{itemize}
		\item Dirigidos a traces
		\begin{itemize}
			\item Comumente usado em sistemas reais
			\item Maior credibilidade
			\item Maior complexidade em simular
			\item A representatividade do trace é fator importante
		\end{itemize}
	\end{itemize}
\end{block}

\end{frame}
%------------------------------------------------

\begin{frame}\frametitle{Simulação vs Emulação}

\begin{block}{Simular ou emular?}
	\begin{itemize}
		
		\item Simulador
		\begin{itemize}
			\item É utilizado para realizar o desenvolvimento e testes de protocolos de rede, especialmente em estágios iniciais de projeto
			\item O custo de simular centenas ou milhares de nós é baixo
			\item A simulação pode ser finalizada em curto tempo
		\end{itemize}

		\item Emulador

		\begin{itemize}
			\item Utiliza firmeware (real) e hardware (simulado/emulado) para desempenhar a simulaçao
			\item Geralmente utilizado em estágios de projeto e validação
			\item O custo de simular centenas ou milhares de nós é alto
			\item O que se desenvolve para um emulador pode ser usado em nós reais
				\begin{itemize}
					\item Fornece mais precisão do desempenho e análises
					\item Maior custo computacional por entidade simulada.
				\end{itemize}
		\end{itemize}

	\end{itemize}
\end{block}

\end{frame}

%------------------------------------------------

\begin{frame}\frametitle{Simuladores}

\begin{block}{Alguns exemplos de simuladores/emuladores}
\begin{itemize}
	\item NS2/NS3
	\item ContikiOS Cooja (simulador e emulador)
	\item TinyOS Tossim
	\item Omnet++
	\item ...
\end{itemize}
\end{block}

\end{frame}

%----------------------------------------------------------------------------------------

\end{document} 
